\section{Figuras}

En esta sección dentro del template, mostraremos algunos ejemplos. En la figura \ref{fig:ejemplo}, creada con el ambiente figure, se presenta una carta de ajuste.

\begin{figure}[htb]
    \centering
    \includegraphics[width=0.55\textwidth,keepaspectratio]{content/images/testimage.png}
    \caption{Carta de ajuste en televisión. Extraida desde \href{https://es.wikipedia.org/wiki/Carta_de_ajuste}{Wikipedia}}
    \label{fig:ejemplo}
\end{figure}


Otra forma de presentar figuras es creando subfiguras como se muestra en la figura \ref{fig:subfiguras}.

\begin{figure}[htb]
    \begin{subfigure}[b]{0.45\linewidth}
        \centering
        \includegraphics[width=0.85\linewidth]{content/images/testimage.png}
        \caption{La subfigura 1}
        \label{fig:subfiguras:a}
    \end{subfigure}
    \begin{subfigure}[b]{0.45\linewidth}
        \centering
        \includegraphics[width=0.85\linewidth]{content/images/testimage.png}
        \caption{La subfigura 2}
        \label{fig:subfiguras:b}
    \end{subfigure}
    \caption{Demostración de subfiguras}
    \label{fig:subfiguras}
\end{figure}

La idea es que podamos hacer referencia a cada una de las subfiguras, la subfigura \ref{fig:subfiguras:a} y la subfigura \ref{fig:subfiguras:b}.