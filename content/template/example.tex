\section{Referencias}
Aqui se muestra un ejemplo de algunas referencias
\cite{godoyNamedEntityRecognition2023, torresPredictingCardiovascularRehabilitation2023}

Mientras que en esta linea, mostramos como citar al autor al hacer mencion de un trabajo \citeauthor{melladoDeepLearningClassifier2023}

\section{Listas y enumeraciones}
En \LaTeX, podemos listar elementos utilizando los comandos itemize y enumerate

\begin{enumerate}
    \item Primer elemento
    \item Segundo elemento
    \item Tercer elemento
\end{enumerate}

\begin{itemize}
    \item Podemos ademas sublistar agregando un itemize dentro de otro itemize
    \begin{itemize}
        \item Primer elemento
        \item Segundo elemento
        \item Tercer elemento
    \end{itemize}
    \item Otro elemento
\end{itemize}


\section{Tablas}
En esta sección, mostraremos un ejemplo de tabla. En la tabla \ref{tab:ejemplo}, se muestra una tabla con tres columnas y tres filas. Por motivos de orden, se recomienda que las tablas sean creadas en archivos separados y luego importadas al documento principal.

\begin{table}
    \centering
    \caption{Ejemplo de tabla}
    \label{tab:ejemplo}
    \begin{tabular}{l S S}
    \toprule
    {Variable} & {Valor} & {Unidad} \\
    \midrule
    {Peso} & \num{70} & \si{\kilogram} \\
    {Altura} & \num{1.7532} & \si{\meter} \\
    {IMC} & \num{22.8} & \si{\kilo\gram\per\meter\squared} \\
    {Valor Incierto} & \num{1.2345\pm0.1234} & \si{\second} \\   
\end{tabular}
\end{table}

La tabla internamente utiliza el paquete booktabs para las lineas de la tabla, y el paquete siunitx para el manejo de unidades.
\texttt{Siunitx} nos permite definir los valores con unidades, y nos permite tambien alinear los valores en la tabla.
Junto a esto, tambien permite redondear los valores de la tabla a una resolución determinada. 
Los valores estan predeterminados en \texttt{preamble/styleconfs/siunitx.tex}


\section{Figuras}
En esta sección dentro del template, mostraremos algunos ejemplos. En la figura \ref{fig:ejemplo}, creada con el ambiente figure, se presenta una carta de ajuste.

\begin{figure}[htb]
    \centering
    \includegraphics[width=0.55\textwidth,keepaspectratio]{content/images/testimage.png}
    \caption{Carta de ajuste en televisión. Extraida desde \href{https://es.wikipedia.org/wiki/Carta_de_ajuste}{Wikipedia}}
    \label{fig:ejemplo}
\end{figure}


Otra forma de presentar figuras es creando subfiguras como se muestra en la figura \ref{fig:subfiguras}.

\begin{figure}[htb]
    \begin{subfigure}[b]{0.45\linewidth}
        \centering
        \includegraphics[width=0.85\linewidth]{content/images/testimage.png}
        \caption{La subfigura 1}
        \label{fig:subfiguras:a}
    \end{subfigure}
    \begin{subfigure}[b]{0.45\linewidth}
        \centering
        \includegraphics[width=0.85\linewidth]{content/images/testimage.png}
        \caption{La subfigura 2}
        \label{fig:subfiguras:b}
    \end{subfigure}
    \caption{Demostración de subfiguras}
    \label{fig:subfiguras}
\end{figure}

La idea es que podamos hacer referencia a cada una de las subfiguras, la subfigura \ref{fig:subfiguras:a} y la subfigura \ref{fig:subfiguras:b}.