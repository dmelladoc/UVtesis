% !TEX encoding=UTF-8
% !TEX program = lualatex
% !TEX spellcheck = en-US

% --------------------------------------------------------------------
% Seleccion de paquetes
% --------------------------------------------------------------------
\DefineCodeSection[true]{BasePkg}
\DefineCodeSection[true]{FontsPkg}
\DefineCodeSection[true]{MathPkg}
\DefineCodeSection[true]{UnitsPkg}
\DefineCodeSection[true]{TablesPkg}
\DefineCodeSection[true]{FiguresPkg}
\DefineCodeSection[true]{TextsPkg}
\DefineCodeSection[true]{BibPkg}
\DefineCodeSection[true]{ListsPkg}
\DefineCodeSection[true]{LayoutPkg}
\DefineCodeSection[true]{HeadFootPkg}
\DefineCodeSection[true]{OtherPkg}
% --------------------------------------------------------------------
% Paquetes básicos
% --------------------------------------------------------------------

\BeginCodeSection{BasePkg}
% Babel para manejo de idiomas
\usepackage{babel}

% Xcolor para manejo de colores
\usepackage[%
    dvipsnames, % Carga una lista de colores predefinidos
    table,      % Permite el uso de colores en tablas
    fixinclude,  % Corrige el problema de includegraphics con xcolor
]{xcolor}

% Graphicx para manejo de imágenes
\usepackage{graphicx}
% Si se carga graphicx, se carga epstopdf
\IfPackageLoaded{graphicx}{%
    \usepackage{epstopdf} % Convierte archivos eps a pdf
}

% Ragged2e para manejo de alineación de texto
\usepackage{ragged2e}
% Varwidth para manejo de anchos variables
\usepackage{varwidth}

\EndCodeSection{BasePkg}

% --------------------------------------------------------------------
% Tipografias
% --------------------------------------------------------------------
\BeginCodeSection{FontsPkg}

% Usaremos Fontspec para manejo de tipografías en luaLaTeX
% Ya que usaremos fuentes externas (la fuente de la Universidad)
\usepackage{fontspec}

\EndCodeSection{FontsPkg}


% --------------------------------------------------------------------
% Matematica
% --------------------------------------------------------------------
\BeginCodeSection{MathPkg}

% Amsmath para manejo de matemáticas
\usepackage[%
    centertags,   % Centra las ecuaciones con varios elementos
    %tbtags,      % Coloca los números de las ecuaciones en la parte superior
    sumlimits,    % Coloca los límites de las sumatorias en la parte superior e inferior
    %nosumlimits, % 
    intlimits,    % Coloca los límites de las integrales en la parte superior e inferior
    %nointlimits, % 
    namelimits,   % Coloca los límites de las funciones en la parte superior e inferior
    %nonamelimits,%
    %leqno,       % Coloca los números de las ecuaciones a la izquierda
    %reqno,       % Coloca los números de las ecuaciones a la derecha
    fleqn,        % Coloca las ecuaciones a la izquierda con identación
]{amsmath}

\IfPackageLoaded{amsmath}{%
    % Cargamos Mathtools para mejoras en amsmath
    \usepackage[fixamsmath,disallowspaces]{mathtools}
}

% Unicode-math para manejo de fuentes matemáticas
\usepackage{unicode-math} 

\EndCodeSection{MathPkg}

% --------------------------------------------------------------------
% Unidades
% --------------------------------------------------------------------
\BeginCodeSection{UnitsPkg}
\usepackage{siunitx}

\EndCodeSection{UnitsPkg}

% --------------------------------------------------------------------
% Tablas
% --------------------------------------------------------------------
\BeginCodeSection{TablesPkg}
% Booktabs para lineas de tablas
\usepackage{booktabs} 

% Multirow para celdas de tablas
% Bigstrut para manejo de celdas grandes
\usepackage{multirow, bigstrut}

% Si se carga hyperref, se carga ltxtable
% Ltxtable para manejo de tablas largas
\ExecuteAfterPackage{hyperref}{%
    \usepackage{ltxtable}
}

\EndCodeSection{TablesPkg}


% --------------------------------------------------------------------
% Figuras
% --------------------------------------------------------------------

\BeginCodeSection{FiguresPkg}

\usepackage{wrapfig} % Para manejo de figuras flotantes

% Evita que las figuras floten entre secciones
\usepackage[section]{placeins}

% Caption permite manejo de leyendas en figuras y tablas
\usepackage{caption}

% Subcaption para manejo de subfiguras
\IfPackageNotLoaded{subfig}{
  \usepackage{subcaption}[2011/08/17]
}

\EndCodeSection{FiguresPkg}

% --------------------------------------------------------------------
% Texto
% --------------------------------------------------------------------
\BeginCodeSection{TextsPkg}
\ExecuteAfterPackage{setspace}{%
    % Footmisc para manejo de notas al pie
    \usepackage[%
        bottom, % Notas al pie al final de la página
        stable, % Notas al pie estables
        perpage, % Reinicia el contador de notas al pie por página
        %para,   % Notas al pie a un lado, en un parrafo
        %side,   % Notas al pie al margen
        ragged,  % Alineación
        %norule, % Sin regla
        multiple,% Notas al pie múltiples
        %symbol, % Notas al pie con símbolos
    ]{footmisc} 
}

\IfDefined{endabstract}{%
    % Abstract para manejo de resumen
    \usepackage{abstract}
}

\EndCodeSection{TextsPkg}

% --------------------------------------------------------------------
% Bibliografia
% --------------------------------------------------------------------
\BeginCodeSection{BibPkg}
% Biblatex para manejo de bibliografía
% TODO: Agregar opciones para definir como vancouver
\IfPackageLoaded{babel}{
    \usepackage{csquotes} % Necesario para biblatex
}
\usepackage[%
    style=vancouver,    % Estilo de la bibliografía
    natbib=true,      % Compatibilidad con natbib
    backend=biber,    % Motor de la bibliografía
    bibwarn=true,     % Advertencias de la bibliografía
    texencoding=auto, % autodetecta la codificación de los archivos
    bibencoding=auto, % autodetecta la codificación de la bibliografía 
]{biblatex}

\EndCodeSection{BibPkg}

% --------------------------------------------------------------------
% Listas, glosarios, etc
% --------------------------------------------------------------------
\BeginCodeSection{ListsPkg}

\ExecuteAfterPackage{hyperref}{%
    % Glossaries para manejo de glosarios
    \usepackage[%
        savewrites,            % Guarda los archivos auxiliares
        translate=true,        % Traduce con babel
        hyperfirst=true,       % Enlace en la primera mención
        %toc,                  % Añade al índice
        numberline,            % Añade número de página
        section=section,       % Presenta por sección
        numberedsection=false, % Por defecto no aparece numerados por seccion
        nonumberlist,          % Sin numeración
        counter=page,          % Numeración por página
        sort=standard,         % Ordenamiento
        acronym,               % Manejo de acrónimos
        shortcuts,             % Atajos (\ac)
    ]{glossaries}
    % Estilos extra de glossaries
    \usepackage{glossary-longragged}    
}

\EndCodeSection{ListsPkg}

% --------------------------------------------------------------------
% Layout
% --------------------------------------------------------------------
\BeginCodeSection{LayoutPkg}
% Geometry para manejo de márgenes
% Dedebemos cargarlo al final para evitar conflictos con otros paquetes
\ExecuteAfterPackage{hyperref}{%
    \usepackage{geometry}
}
\ExecuteAfterPackage{lastpackage}{%
    \IfPackageNotLoaded{geometry}{%
        \usepackage{geometry}
    }
}

\EndCodeSection{LayoutPkg}

% --------------------------------------------------------------------
% Encabezados y pies de página
% --------------------------------------------------------------------

\BeginCodeSection{HeadFootPkg}
\usepackage[%
    automark,                 % Automatiza la marca de secciones
    autooneside,              % Encabezado en una sola cara
    pagestyleset=KOMA-Script, % Estilo de página
    markcase=ignoreuppercase, % Ignora mayúsculas
]{scrlayer-scrpage}

\EndCodeSection{HeadFootPkg}


% --------------------------------------------------------------------
% Otros paquetes
% --------------------------------------------------------------------

\BeginCodeSection{OtherPkg}
% Hyperref para manejo de enlaces
\usepackage[%
    backref=page,        % Añade enlace a la página de la cita
    pagebackref=false,    %
    hyperindex=true,     % Añade enlace en el índice
    hyperfootnotes=true, % Añade enlace en las notas al pie
    bookmarks=true,      % 
    pdfpagelabels=true,  % set PDF page labels 
]{hyperref}

% Bookmark para manejo de marcadores para hyperref
\IfNotDraft{%
  \usepackage{bookmark}
}

\usepackage[htt]{hyphenat}

\EndCodeSection{OtherPkg}

% --------------------------------------------------------------------
% Para finalizar, detectamos el ultimo paquete con lastpackage
% --------------------------------------------------------------------
\usepackage{lastpackage}